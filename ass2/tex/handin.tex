\documentclass[a4paper,10pt]{article}
\usepackage[a4paper, total={210mm,297mm}, left=20mm, right=20mm, top=20mm, bottom=20mm]{geometry}
\usepackage[utf8]{inputenc}
\usepackage{amsmath}
\usepackage{parskip}
\renewcommand\thesection{\Roman{section}}
\renewcommand\thesubsection{\Roman{section}.\arabic{subsection}}
\renewcommand\thesubsubsection{\Roman{section}.\arabic{subsection}.\alph{subsubsection}}

%opening
\title{Assignment 2  \\Programming Massively Parallel Hardware }
\author{Mads Thoudahl /qmh332}

\begin{document}

\maketitle

To improve readability of my solutions, some extra pages have been used although not utilized to its capacity.
Sorry for the late handin, but i wanted to be able to include any reporting issues pointed out in the feedback from the first assignment... And as i have small children, not all of my weekend can be reserved..



\vfill
\section{CUDA Programming with reduce and Segmented Scan}


\vfill
\subsection{Task 1 Implement ExclusiveScan and SegmentedExclusiveScan}
By using the already implemented \texttt{ScanInclusive} code, the algorithm will look something like this:

\begin{verbatim}
 ScanExc(arr_in):
   arr_tmp       <- ScanInc OP       # [a1, a1 OP a2, ..., a1 OP .. OP an ]
   arr_out[1:n]  <- arr_tmp[0:n-1]   # Shift Right operation   - O(N/P)
   arr_out[0]    <- neutral element  # insert neutral element  - O(N/P)
   return arr_out
\end{verbatim}

The attached code compiles and runs, and does indeed test the (segmented)ScanExclusive implementation.
\vfill

\newpage

\subsection{Task 2 - Implement MaximumSegmentSum problem}
Algorithm implemented:
\begin{verbatim}
 ScanExc(arr_in):
   arr_myint4    <- Conversion(arr_in) # [(a1,a1,a1,a1),..,(an,an,an,an)]
   arr_tmp       <- ScanInc MsspOp     # [A1, A1 OP A2, ..., A1 OP .. OP An ]
   res_myint4    <- arr_tmp[last]      # A1 OP A2 OP .. OP An
   res           <- choose(res_myint4) # (res, _mis, _mcs, _ts)
   return res
\end{verbatim}
As I had a bit of inconvenience using scalar int type, while implementing, I chose to 'get on' returning an entire array with a single value (i know that is a waste, i will correct if it has any impact on score).

\begin{verbatim}
from ScanKernels.cu.h
msspTrivialMap(int* inp_d, MyInt4* inp_lift, int inp_size) {
    const unsigned int gid = blockIdx.x*blockDim.x + threadIdx.x;
    if(gid < inp_size) {
        int x = inp_d[gid];
        if ( x > 0 ) inp_lift[gid] = MyInt4 ( x, x, x, x);
        else inp_lift[gid] = MyInt4 ( 0, 0, 0, x);
    }
}

class MsspOp {
  public: ..
    static __device__ inline MyInt4 apply(volatile MyInt4& t1, volatile MyInt4& t2) {
        int mss = max( max( t1.x, t2.x ),( t1.z + t2.y )); // max segment sum
        int mis = max( t1.y, ( t1.w + t2.y ));             // max initial sum
        int mcs = max(( t1.z + t2.w ), t2.z);              // max conclusive sum
        int t   = (t1.w+ t2.w);                            // total sum
        return MyInt4(mss, mis, mcs, t);    }     
};
\end{verbatim}
See attached code for further insights
Code runs and successfully tests the quite boring array [1,1,1,1,..] where \texttt{mss=n} and n is length of array.


\newpage
\subsection{Task 3 - Implement Sparse Vector Matrix Multiplication in CUDA}

\subsubsection{Nested implementation in haskell}
\begin{verbatim}
nestSparseMatVctMult :: [[(Int,Double)]] -> [Double] -> [Double]
nestSparseMatVctMult mat vec =
    map (\row -> let (idxs,vals) = unzip row
                     vvals       = map (\idx ->  vec !! idx) idxs
                     prods       = map (\(x,y)-> x*y) (zip vals vvals)
                     res         = scanInc (+) 0 prods
                 in  last res
        ) mat
\end{verbatim}

This Piece of code is tested and works on my own unstructured tests and correctly calculates the matrix in the \texttt{main} function of \texttt{PrimesQuickSort.hs}.

\subsubsection{flattened implementation in haskell}
In the \texttt{vvals} and \texttt{prods} we have a map inside a map, easily resolved by the 2nd rule.
in the \texttt{res}, the scan is converted to a segmented scan, no problemo..
The problem appears on retrieving the last result, when it is hidden inside a list of segments.
\begin{verbatim}
flatSparseMatVctMult :: [Int] -> [(Int,Double)] -> [Double] -> [Double]
flatSparseMatVctMult flags mat vec =                           -- flags   [1,0,1,0,0,1,0,0,0]
  let tot_num_elems = length flags
      vlen          = length vec
      (idxs, vals)  = unzip mat
      result        = replicate (last $ scanInc (+) 0 flags) 0
      vvals         = map (\idx -> vec !! idx) idxs
      prods         = map (\(val, vval) -> val * vval) (zip vals vvals)
      rowsizes      = zipWith (*) flags (scanInc (+) 0 flags)  -- goal    [2,0,3,0,0,4,0,0,0] - failed!
      sgmres        = segmScanInc (+) 0 rowsizes prods         -- resuls  [_,r,_,_,r,_,_,_,r]
      matchidx      = [0..]                                    -- matchidx[0,1,2,3,4,5,6,7,8]
      residx        = map (\x -> x-1) (scanInc (+) 0 rowsizes) -- res idx [1,1,4,4,4,8,8,8,8]
      newidx        = map (\x -> x-1) (scanInc (+) 0 flags)    -- new idx [0,0,1,1,1,2,2,2,2]
      _write        = map (\(rres,mi,ri,ni) -> if (mi==ri) then write [ni] [rres] result
                                                           else write  [0] [rres]  prods ) -- dummy, but else required
      -- would use if/else here but last overwrite fixes this, will use if/else in cuda version!
             (zip4 sgmres matchidx residx newidx)
     in  result
\end{verbatim}
Some difficulties making the last part compile, writing to an array, but the idea should be sound.

The challenge i have not been able to overcome is converting the \texttt{flags} list to a \texttt{sizes} list e.g. [1,0,1,0,0,1,0,0,0] $\rightarrow$ [2,0,3,0,0,4,0,0,0], having the \texttt{rowsizes} failing, thus missing that part in succeeding.

\newpage
\subsubsection{flattened implementation in cuda}
Some important parts of the code is presented here, the code in its entirety is attached.

As it turns out, all the map operations of finding out which vector indices and what matrice entries to multiply them with are resolved in \texttt{spMatVctMult\_pairs} function below.
The summation (reduction) is produced by a segmented inclusive scan op:(+) ne:$0$ function, handling all Matrix columns in parallel, very smooth. Last part sorts out where the results are hiding and digs them out of the scanned array.
\begin{verbatim}
spMatVctMult_pairs(int* mat_inds, float* mat_vals, float* vct, int tot_size, float* tmp_pairs) {
    const unsigned int gid = blockIdx.x*blockDim.x + threadIdx.x;
    if(gid < tot_size) {
        tmp_pairs[gid] = vct[ mat_inds[gid] ] * mat_vals[gid];
    }
}

write_lastSgmElem(float* tmp_scan, int* tmp_inds, int* flags_d, int tot_size, float* vct_res) {
    const unsigned int gid = blockIdx.x*blockDim.x + threadIdx.x;
    if(gid < tot_size) {
        if ( flags_d[gid+1] != 0 )     // all but the last dotproduct
            vct_res[tmp_inds[gid]-1] = tmp_scan[gid];
    } else if (gid == (tot_size-1)) {  // the last dotproduct
            vct_res[tmp_inds[gid]-1] = tmp_scan[gid];
    }
}
\end{verbatim}
The writing back to the correct locations is far easier in the imperative cuda language than in Haskell, which caused the trouble in the previous part of this task. The implementation of \texttt{write\_lastSgmElem} fixes it.

\begin{tabular}{l|cc|r}
The three steps & Depth & Work & Comment\\ \hline
multiplication of pairs into product array & $\mathcal{O}(N)$ & $\mathcal{O}(N)$ & some memory transactions may be improved \\
segmented scan of product array & $\mathcal{O}(N)$ & $\mathcal{O}(N)$ & \\
locating and writing back results & $\mathcal{O}(P)$ & $\mathcal{O}(N)$ & easier than haskell \\ \hline
\end{tabular}

Output when run
\begin{verbatim}
 ./sgmScanInc
sScan Inclusive on GPU runs in: 1371459 microsecs
sScan Inclusive +   VALID RESULT!
 Scan Inclusive on GPU runs in: 8395 microsecs
 Scan Inclusive +   VALID RESULT!
sScan Inclusive on GPU runs in: 8108 microsecs
sScan Inclusive +   VALID RESULT!
 Scan Exclusive on GPU runs in: 7846 microsecs
 Scan Exclusive +   VALID RESULT!
sScan Exclusive on GPU runs in: 8202 microsecs
sScan Exclusive +   VALID RESULT!
MaximumSegmentSum on GPU runs in: 4549 microsecs
MSS +   VALID RESULT! 987624
Performing sparse Matrix vector multiplicationsparse Matrix-Vector multiplication on GPU runs in: 302 microsecs
SparseMatrix Vector Multiplication Result:
[ 3.000000 0.000000 -4.000000 6.000000 ] - calculation
[ 3.000000 0.000000 -4.000000 6.000000 ] - correct result
\end{verbatim}
The results are tested, and seems to work, so proof of concept... but no timing and comparison of sequential implementation of the like of a matrix multiplication application in this handin.

A bit odd that the first run (scaninc) allways seem very slow.. is it normal?? some initialization problem? 



\newpage

\section{Hardware Track Exercises}
\subsection{Task 1 - 5-Stage pipeline design}
In all the subtasks I am assuming (knowing it is a modified truth) that all types of branches uses 'one' instruction to perform..

\subsubsection{No forwarding, No hazard detection, rewrite code}
\begin{verbatim}
          OPERATION         |  COMMENT
 ________________________________________________________________________________________________
 SEARCH:  LW   R5 0(R3)     |
          NOOP              |  RAW HAZARD - R5
          NOOP              |  RAW HAZARD - R5
          NOOP              |  RAW HAZARD - R5
          SUB  R6 R5 R2     |
          NOOP              |  RAW HAZARD - R6
          NOOP              |  RAW HAZARD - R6
          NOOP              |  RAW HAZARD - R6
          BNEZ R6 NOMATCH   |  Branch?? Considerations of consequences below
          ADDI R1 R1 #1     |  On branch, simply omit writeback to R1, no harm done
 NOMATCH: ADDI R3 R3 #4     |  On branch, this instruction will be fetched more than once,
                            |  but omitting WB to R3 in first execution will remedy the situation
          NOOP              |  RAW HAZARD - R3
          NOOP              |  RAW HAZARD - R3
          NOOP              |  RAW HAZARD - R3
          BNE  R4 R3 SEARCH |
\end{verbatim}

\newpage
\subsubsection{No forwarding, Hazard detection stalls, count clockcycles used}

17 Clockcycles are used on a matching word.
\begin{verbatim}
 CLK    (OPERATION)/PHASES         |  COMMENT   HDU:HazardDetectionUnit
 ________________________________________________________________________________________________
    (LW)  
 1   IF (SUB)
 2   ID  IF (BNEZ) 
 3   EX  ID  IF                           | HDU reports hazard on R5 and stalls
 4   MEM ID  IF     
 5   WB  ID  IF 
 6       ID  IF (ADDI)
 7       EX  ID  IF                       | HDU reports hazard on R6 and stalls
 8       MEM ID  IF    
 9       WB  ID  IF  
10           ID  IF (ADDI)
11           EX  ID  IF (BNE)
12           MEM EX  ID  IF (??)          | HDU reports hazard on R3 and stalls
13           WB  MEM EX  ID  IF           
14               WB  MEM ID  ..   
15                   WB  ID  .. 
16                       ID  .. (??)      | both ?? ops are skipped due to repeat of search
17                       EX  ..  IF (LW)  | Repeats search...
18                       MEM ..  ..  IF   
\end{verbatim}



18 Clockcycles are used on a word not matching.
\begin{verbatim}
 CLK    (OPERATION)/PHASES         |  COMMENT    |.. means WB disabled
 ________________________________________________________________________________________________
 0  (LW)  
 1   IF (SUB)
 2   ID  IF (BNEZ) 
 3   EX  ID  IF                           | HDU reports hazard on R5 and stalls
 4   MEM ID  IF     
 5   WB  ID  IF 
 6       ID  IF (ADDI)
 7       EX  ID  IF                       | HDU reports hazard on R6 and stalls
 8       MEM ID  IF    
 9       WB  ID  IF  
10           ID  IF (ADDI)
11           EX  ID  IF (ADDI)            | BNEZ JUMP! to address of 2nd ADDI instruction
12           MEM ..  ..  IF (BNE)
13           WB  ..  ..  ID  IF (??)      | 
14               ..  ..  EX  ID  IF       | HDU reports hazard on R3 and stalls
15                   ..  MEM ID  ..  
16                       WB  ID  .. 
17                           ID  .. (??)
18                           EX  ..  IF (LW)    | Repeats search
19                           MEM ..  ..  IF
\end{verbatim}

\newpage
\subsubsection{Assuming internal register forwarding}
I assume that internal register forwarding means that the values of a given register currently written is available to next ID in the same cycle as WB..

14 Clockcycles are used on a matching word.
\begin{verbatim}
 CLK    (OPERATION)/PHASES         |  COMMENT   HDU:HazardDetectionUnit
 ________________________________________________________________________________________________
    (LW)  
 1   IF (SUB)
 2   ID  IF (BNEZ) 
 3   EX  ID  IF                           | HDU reports hazard on R5 and stalls
 4   MEM ID  IF     
 5   WB  ID  IF (ADDI)
 6       EX  ID  IF                       | HDU reports hazard on R6 and stalls
 7       MEM ID  IF    
 8       WB  ID  IF (ADDI) 
 9           EX  ID  IF (BNE)
10           MEM EX  ID  IF (??)          | HDU reports hazard on R3 and stalls
11           WB  MEM EX  ID  IF           
12               WB  MEM ID  ..   
13                   WB  ID  .. (??)      | both ?? ops are skipped due to repeat of search
14                       EX  ..  IF (LW)  | Repeats search...
15                       MEM ..  ..  IF   
\end{verbatim}


15 Clockcycles are used on a word not matching.
\begin{verbatim}
 CLK    (OPERATION)/PHASES         |  COMMENT    |.. means WB disabled
 ________________________________________________________________________________________________
 0  (LW)  
 1   IF (SUB)
 2   ID  IF (BNEZ) 
 3   EX  ID  IF                           | HDU reports hazard on R5 and stalls
 4   MEM ID  IF     
 5   WB  ID  IF (ADDI)
 6       EX  ID  IF                       | HDU reports hazard on R6 and stalls
 7       MEM ID  IF    
 8       WB  ID  IF (ADDI)
 9           EX  ID  IF (ADDI)            | BNEZ JUMP! to address of 2nd ADDI instruction
10           MEM ..  ..  IF (BNE)
11           WB  ..  ..  ID  IF (??)      | 
12               ..  ..  EX  ID  IF       | HDU reports hazard on R3 and stalls
13                   ..  MEM ID  ..  
14                       WB  ID  .. (??)
15                           EX  ..  IF (LW)    | Repeats search
16                           MEM ..  ..  IF
\end{verbatim}




\newpage

\subsubsection{Assuming full forwarding and hazard detection}

10 Clockcycles are used on a matching word. Additional 2 clocks are used after last search to complete writeback phase.
\begin{verbatim}
 CLK    (OPERATION)/PHASES         |  COMMENT   HDU:HazardDetectionUnit
 ________________________________________________________________________________________________
    (LW)  
 1   IF (SUB)
 2   ID  IF (BNEZ) 
 3   EX  ID  IF                           | HDU reports hazard on R5 and stalls
 4   MEM ID  IF     
 5   WB  ID  IF (ADDI)
 6       EX  ID  IF (ADDI)                | Forwarding value
 7       MEM EX  ID  IF (BNE)             | Forwarding value
 8       WB  MEM EX  ID  IF (??)
 9           WB  MEM EX  ID  IF (??)      | both ?? ops are skipped due to repeat of search
10               WB  MEM EX  ..  IF (LW)  | Repeats search...
11                   WB  MEM ..  ..  IF    
12                       WB  ..  ..
\end{verbatim}


11 Clockcycles are used on a word not matching.
\begin{verbatim}
 CLK    (OPERATION)/PHASES         |  COMMENT   HDU:HazardDetectionUnit
 ________________________________________________________________________________________________
    (LW)  
 1   IF (SUB)
 2   ID  IF (BNEZ) 
 3   EX  ID  IF                               | HDU reports hazard on R5 and stalls
 4   MEM ID  IF     
 5   WB  ID  IF (ADDI)                    
 6       EX  ID  IF (ADDI)                    | Forwarding value
 7       MEM EX  ID  IF (ADDI)                | BNEZ JUMP! to address of 2nd ADDI instruction
 8       WB  MEM ..  ..  IF (BNE)
 9           WB  ..  ..  ID  IF (??)      
10               ..  ..  EX  ID  IF (??)      | fwd R3 to BNE
11                   ..  MEM EX  ..  IF (LW)  | both ?? ops are skipped due to repeat of search
12                       WB  MEM ..  ..  IF   | Repeats search...
\end{verbatim}

\subsubsection{improvements}
By unrollin the loop, more values can be loaded into the registers, avoiding the hazards, thus avoid stalling - minimum 2 more registers are needed.
As branching takes more time, make the common case faster by rewriting the code to branch in the lesser common case.. i.e. on a match..
If search area is not divisible by \# regs used in loop unroll, some code has to be included to correctly handle the end-case.
Delayed branching will not help much for this chunk of code, as we have got 2 branches in 6 instructions, thus not enough instructions to fill in the 2 slots behind the branches.
If one performs a software pipelining, one just might be able to utilize the 2 slots in the delayed branching, But it will have an impact on the amount of used registers.


\newpage
\subsection{Task 2 - Vector Machines}
\subsubsection{The machine code with vector instructions for each 64 bit slice using loads and arithmetic instructions}
We have 8 vector registers, performing 2 operations will be possible at the same time as both use 3.
This assumes that there exist 2 MUL components..

I believe the code underneath produces a 2 vector ops in parallel, reducing the needed loops to 8 (instead of 16).
\begin{verbatim}
        OPERATION                |  COMMENT
 __________________________________________________________________________________________
        ADDI   R5    R0    #1024 | Set length of calculation
        ADDI   R4    R0    #1    | Set Register4 to 1 (the stride) 
 LOOP:                           | ASSUME 2 MUL COMPONENTS
        L.V    V1    0(R1) R4    | Load from pos in Reg1 by the stride of 1 into Vreg1
        L.V    V2    0(R2) R4    | Load from pos in Reg2 by the stride of 1 into Vreg2
        ADDI   R1    R1    #512  | update address for R1 for next iteration
        ADDI   R2    R2    #512  |  --||-- for R2 (assuming length of '8' Bytes - doubles)
        L.V    V3    0(R1) R4    | Load from pos in Reg1 by the stride of 1 into Vreg3
        L.V    V4    0(R2) R4    | Load from pos in Reg2 by the stride of 1 into Vreg4
        ADDI   R1    R1    #512  | update address for R1 for next iteration
        ADDI   R2    R2    #512  |  --||-- for R2 (assuming length of '8' Bytes - doubles)

        MUL.V  V5    V1    V2    | V5 <- V1 * V2   elementwise
        MUL.V  V6    V1    V2    | V6 <- V3 * V4   elementwise

        S.V    V5    0(R3)       | Store to mem starting at pos at R3
        ADDI   R3    R3    #512  | Update address for R3 (store address)
        S.V    V6    0(R3)       | Store to mem starting at pos at R3
        ADDI   R3    R3    #512  | Update address for R3 (store address)

        SUBBI  R5    R5    #128  | Update outer loop index counter  by 2*64  
        BNEZ   R5    LOOP        | Repeat until end of calc
\end{verbatim}


\subsubsection{Compute the time taken (in clocks) by a dot product of size 1024}
Using the code above we get:
$$ T_{execution} = startup(load) + startup(mul) + startup store + Number of vectors * Vector Length $$
$$ T_{execution} = 30 + 10 + 30 + (16/2) * 64  = 582 $$
I neglect the few instructions might be involved by duplicating the code (estimated 5-8 clc), and the time used to sum up the stored vector.


\subsubsection{Compute the time taken (in clocks) by a matrix multiplication of two 1024 x 1024 matrices}
Each entrance of the resulting matrix is created by a dotproduct of two 1024 vectors. as there exist $1024^2$ entrances, each taking $582$ clockcycles to calculate (neglecting the addition phase).
$$T_{mmul} = 582 \cdot 1024^2 \sim 610.3 M clocks $$
This assumes the matrices are represented as dense matrices, which each takes up $8*1024*1024 \sim 8 M $ Bytes of data.

\end{document}
