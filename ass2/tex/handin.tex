\documentclass[a4paper,10pt]{article}
\usepackage[a4paper, total={210mm,297mm}, left=20mm, right=20mm, top=20mm, bottom=20mm]{geometry}
\usepackage[utf8]{inputenc}
\usepackage{amsmath}
\usepackage{parskip}
\renewcommand\thesection{\Roman{section}}
\renewcommand\thesubsection{\Roman{section}.\arabic{subsection}}
\renewcommand\thesubsubsection{\Roman{section}.\arabic{subsection}.\alph{subsubsection}}

%opening
\title{Assignment 2  \\Programming Massively Parallel Hardware }
\author{Mads Thoudahl /qmh332}

\begin{document}

\maketitle

To improve readability of my solutions, some extra pages have been used although not utilized to its capacity.

\section{CUDA Programming with reduce and Segmented Scan}
\subsection{Task 1 Implement ExclusiveScan and SegmentedExclusiveScan}
See attached code

\subsection{Task 2 - Implement MaximumSegmentSum problem}
See attached code


\subsection{Task 3 - Implement Sparse Vector Matrix Multiplication in CUDA}
See attached code in files \texttt{sparsevectmult.cu}

\newpage

\section{Hardware Track Exercises}
\subsection{Task 1 - 5-Stage pipeline design}
In all the subtasks I am assuming (knowing it is a modified truth) that all types of branches uses 'one' instruction to perform..

\subsubsection{No forwarding, No hazard detection, rewrite code}
\begin{verbatim}
          OPERATION         |  COMMENT
 ________________________________________________________________________________________________
 SEARCH:  LW   R5 0(R3)     |
          NOOP              |  RAW HAZARD - R5
          NOOP              |  RAW HAZARD - R5
          NOOP              |  RAW HAZARD - R5
          SUB  R6 R5 R2     |
          NOOP              |  RAW HAZARD - R6
          NOOP              |  RAW HAZARD - R6
          NOOP              |  RAW HAZARD - R6
          BNEZ R6 NOMATCH   |  Branch?? Considerations of consequences below
          ADDI R1 R1 #1     |  On branch, simply omit writeback to R1, no harm done
 NOMATCH: ADDI R3 R3 #4     |  On branch, this instruction will be fetched more than once,
                            |  but omitting WB to R3 in first execution will remedy the situation
          NOOP              |  RAW HAZARD - R3
          NOOP              |  RAW HAZARD - R3
          NOOP              |  RAW HAZARD - R3
          BNE  R4 R3 SEARCH |
\end{verbatim}

\newpage
\subsubsection{No forwarding, Hazard detection stalls, count clockcycles used}

17 Clockcycles are used on a matching word.
\begin{verbatim}
 CLK    (OPERATION)/PHASES         |  COMMENT   HDU:HazardDetectionUnit
 ________________________________________________________________________________________________
    (LW)  
 1   IF (SUB)
 2   ID  IF (BNEZ) 
 3   EX  ID  IF                           | HDU reports hazard on R5 and stalls
 4   MEM ID  IF     
 5   WB  ID  IF 
 6       ID  IF (ADDI)
 7       EX  ID  IF                       | HDU reports hazard on R6 and stalls
 8       MEM ID  IF    
 9       WB  ID  IF  
10           ID  IF (ADDI)
11           EX  ID  IF (BNE)
12           MEM EX  ID  IF (??)          | HDU reports hazard on R3 and stalls
13           WB  MEM EX  ID  IF           
14               WB  MEM ID  ..   
15                   WB  ID  .. 
16                       ID  .. (??)      | both ?? ops are skipped due to repeat of search
17                       EX  ..  IF (LW)  | Repeats search...
18                       MEM ..  ..  IF   
\end{verbatim}



18 Clockcycles are used on a word not matching.
\begin{verbatim}
 CLK    (OPERATION)/PHASES         |  COMMENT    |.. means WB disabled
 ________________________________________________________________________________________________
 0  (LW)  
 1   IF (SUB)
 2   ID  IF (BNEZ) 
 3   EX  ID  IF                           | HDU reports hazard on R5 and stalls
 4   MEM ID  IF     
 5   WB  ID  IF 
 6       ID  IF (ADDI)
 7       EX  ID  IF                       | HDU reports hazard on R6 and stalls
 8       MEM ID  IF    
 9       WB  ID  IF  
10           ID  IF (ADDI)
11           EX  ID  IF (ADDI)            | BNEZ JUMP! to address of 2nd ADDI instruction
12           MEM ..  ..  IF (BNE)
13           WB  ..  ..  ID  IF (??)      | 
14               ..  ..  EX  ID  IF       | HDU reports hazard on R3 and stalls
15                   ..  MEM ID  ..  
16                       WB  ID  .. 
17                           ID  .. (??)
18                           EX  ..  IF (LW)    | Repeats search
19                           MEM ..  ..  IF
\end{verbatim}

\newpage
\subsubsection{Assuming internal register forwarding}
I assume that internal register forwarding means that the values of a given register currently written is available to next ID in the same cycle as WB..

14 Clockcycles are used on a matching word.
\begin{verbatim}
 CLK    (OPERATION)/PHASES         |  COMMENT   HDU:HazardDetectionUnit
 ________________________________________________________________________________________________
    (LW)  
 1   IF (SUB)
 2   ID  IF (BNEZ) 
 3   EX  ID  IF                           | HDU reports hazard on R5 and stalls
 4   MEM ID  IF     
 5   WB  ID  IF (ADDI)
 6       EX  ID  IF                       | HDU reports hazard on R6 and stalls
 7       MEM ID  IF    
 8       WB  ID  IF (ADDI) 
 9           EX  ID  IF (BNE)
10           MEM EX  ID  IF (??)          | HDU reports hazard on R3 and stalls
11           WB  MEM EX  ID  IF           
12               WB  MEM ID  ..   
13                   WB  ID  .. (??)      | both ?? ops are skipped due to repeat of search
14                       EX  ..  IF (LW)  | Repeats search...
15                       MEM ..  ..  IF   
\end{verbatim}


15 Clockcycles are used on a word not matching.
\begin{verbatim}
 CLK    (OPERATION)/PHASES         |  COMMENT    |.. means WB disabled
 ________________________________________________________________________________________________
 0  (LW)  
 1   IF (SUB)
 2   ID  IF (BNEZ) 
 3   EX  ID  IF                           | HDU reports hazard on R5 and stalls
 4   MEM ID  IF     
 5   WB  ID  IF (ADDI)
 6       EX  ID  IF                       | HDU reports hazard on R6 and stalls
 7       MEM ID  IF    
 8       WB  ID  IF (ADDI)
 9           EX  ID  IF (ADDI)            | BNEZ JUMP! to address of 2nd ADDI instruction
10           MEM ..  ..  IF (BNE)
11           WB  ..  ..  ID  IF (??)      | 
12               ..  ..  EX  ID  IF       | HDU reports hazard on R3 and stalls
13                   ..  MEM ID  ..  
14                       WB  ID  .. (??)
15                           EX  ..  IF (LW)    | Repeats search
16                           MEM ..  ..  IF
\end{verbatim}




\newpage

\subsubsection{Assuming full forwarding and hazard detection}

10 Clockcycles are used on a matching word. Additional 2 clocks are used after last search to complete writeback phase.
\begin{verbatim}
 CLK    (OPERATION)/PHASES         |  COMMENT   HDU:HazardDetectionUnit
 ________________________________________________________________________________________________
    (LW)  
 1   IF (SUB)
 2   ID  IF (BNEZ) 
 3   EX  ID  IF                           | HDU reports hazard on R5 and stalls
 4   MEM ID  IF     
 5   WB  ID  IF (ADDI)
 6       EX  ID  IF (ADDI)                | Forwarding value
 7       MEM EX  ID  IF (BNE)             | Forwarding value
 8       WB  MEM EX  ID  IF (??)
 9           WB  MEM EX  ID  IF (??)      | both ?? ops are skipped due to repeat of search
10               WB  MEM EX  ..  IF (LW)  | Repeats search...
11                   WB  MEM ..  ..  IF    
12                       WB  ..  ..
\end{verbatim}


11 Clockcycles are used on a word not matching.
\begin{verbatim}
 CLK    (OPERATION)/PHASES         |  COMMENT   HDU:HazardDetectionUnit
 ________________________________________________________________________________________________
    (LW)  
 1   IF (SUB)
 2   ID  IF (BNEZ) 
 3   EX  ID  IF                               | HDU reports hazard on R5 and stalls
 4   MEM ID  IF     
 5   WB  ID  IF (ADDI)                    
 6       EX  ID  IF (ADDI)                    | Forwarding value
 7       MEM EX  ID  IF (ADDI)                | BNEZ JUMP! to address of 2nd ADDI instruction
 8       WB  MEM ..  ..  IF (BNE)
 9           WB  ..  ..  ID  IF (??)      
10               ..  ..  EX  ID  IF (??)      | fwd R3 to BNE
11                   ..  MEM EX  ..  IF (LW)  | both ?? ops are skipped due to repeat of search
12                       WB  MEM ..  ..  IF   | Repeats search...
\end{verbatim}

\subsubsection{improvements}
By unrollin the loop, more values can be loaded into the registers, avoiding the hazards, thus avoid stalling - minimum 2 more registers are needed.
As branching takes more time, make the common case faster by rewriting the code to branch in the lesser common case.. i.e. on a match..
If search area is not divisible by # regs used in loop unroll, some code has to be included to correctly handle the end-case.

\newpage
\subsection{Task 2 - Vector Machines}
Tuesday lecture

\end{document}
